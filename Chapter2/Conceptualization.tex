\chapter[Conceptualization]{Conceptualisation}
\label{Chap:Conceptualisation}

This chapter establishes the theoretical foundation for understanding the complex role of collective memory in populist and illiberal political contexts. It explores how memory is not simply a passive record of the past but an active and contested terrain where political actors, especially right-wing populists, construct narratives that reshape national identity and legitimise power. Central concepts such as mnemonic warfare, othering, and the role of media in shaping and disseminating these narratives are examined.

\section{Collective Memory}

In this thesis, collective memory is viewed not as a passive storage of facts, but as an active, social process in which group identity is constantly formed and justified. It is not a neutral or objective space, but a contested one where different stories compete for acceptance and dominance. Historical interpretation is employed to construct identity and acquire political power \citep{kapralski_battlefields_2001}.

Collective memory encompasses symbolic myths and cultural traumas that constitute a nation's identity. Mythic narratives play the role of binding communities together through shared interpretations of victimhood, which have the effect of legitimising current political agendas \citep{stanczyk_commemorating_2014}. Monuments, museums, and other material sites are central to this conflict, as they frame and physically inscribe memory in ways that echo and affirm existing power relations and group identities \citep{kapralski_battlefields_2001}.

Memory is not only shaped by what is remembered but also by what is forgotten. Collective forgetting can occur when institutions collapse, historical discourses become irrelevant due to shifting sociopolitical conditions, or mnemonic practices are no longer actively reproduced, exposing memory to erosion and extinction \citep{foroughi_collective_2020}. Just as memory can be constructed strategically, silence can also be strategically designed. Forgetting politics that render histories unspoken, unmarked, or denied is also key to shaping public perceptions of the past. Being able to dictate what is omitted from collective memory speaks to underlying recognition and legitimacy hierarchies \citep{vinitzky-seroussi_unpacking_2010}. This does not occur passively but is instead actively maintained and manipulated for contemporary ideological goals \citep{stanczyk_commemorating_2014}. This loss of "mnemonic infrastructure" can result in loss of belonging and identity \citep{foroughi_collective_2020}.

Memory is not stored in a void, nor is it recalled solely on an individual level. It is shaped by discursive practices, such as media, political speeches, education, and commemorative rituals, which actively select, emphasise, or omit certain aspects of the past. Furthermore, memory is framed by generational and cultural influences \citep{wydra_generations_2018}. Building on the early work of \citet{halbwachs_space_1980}, even seemingly personal memory is mediated through social structures. Memory is always qualified by collective activity and shared milieux  \citep{wertsch_narrative_2008}.

One key insight here is \posscite{nora_between_1989} conceptualisation of "\textit{lieux de mémoire}" or "sites of memory", which highlights how national memory is institutionally and symbolically grounded in public life \citep{olick_social_1998}. Nowadays, social media is seen as one of these sites, and research on memory sites has proliferated exponentially since the 1960s, greatly expanding our understanding of collective memory \citep{wang_internet_2025}.

Moving away from fixed or sacred views of tradition, the field now understands that collective memory is constructed and often utilised for specific purposes. Following Foucault’s approach, scholars have challenged the notion that historical narratives are inherently "true", demonstrating how they are inextricably linked to knowledge and power \citep{olick_social_1998}. Recent research also shows that collective memory is often shaped into state-approved stories. These systems extend beyond the state; they encompass networks of experts, media, and cultural institutions that help shape and legitimise specific versions of history \citep{dujisin_reassessing_2024}. Although the term "collective memory" has been faulted as metaphorical and conceptually imprecise \citep{gedi_collective_1996}, it remains a useful analytical concept.

Briefly, this thesis adopts a constructivist and critical stance towards collective memory as a contested and politicised process that is active. Memory is neither naturally stored nor recalled by individuals themselves, but instead socially constructed and framed through discursive routines and generational continuities \citep{wydra_generations_2018}.  Scholars like Wertsch have sharpened their use by distinguishing between specific narrative content and more abstract narrative schemas. This enables an easier analysis of how memory functions in political and cultural arenas \citep{wertsch_narrative_2008}.

\section{The Politics of Memory}

This subsection examines how collective memory is constructed, contested, and instrumentalised within political contexts. It first considers the role of narratives and framing (\hyperref[sec:narrative_and_framing]{2.2.1}), then explores memory as a form of power embedded in institutions and transnational dynamics (\hyperref[sec:memory_is_power]{2.2.2}) and finally analyses the strategies of mnemonic warriors and memory laws in consolidating official histories (\hyperref[sec:mnemonic_warriors_and_memory_laws]{2.2.3}).

\subsection{Narratives and Framing}
\label{sec:narrative_and_framing}

Narrative is a key mechanism through which collective memory is organised. A straightforward narrative that lends emotional significance and moral importance to historical events is created. Society passes these memory narratives through schools, politics, and the media, where they are often turned into simplified myths that reduce the complexity of the past \citep{smorti_why_2016}. Narrative control is a key goal of political communication. Historical metaphors are used to create links in the public's mind, evoking imagery and emotion. This often leads to historical reductionism and oversimplification \citep{mumford_parallels_2015}.

Framing is another key mechanism through which collective memory is organised. Framing, originally a concept from the journalistic field, refers to the process of shaping public perception of events by emphasising some aspects while downplaying or omitting others. When applied to history, historical framing involves the selective presentation of past events in ways that influence collective understanding, interpretation, and memory. Through historical framing, narratives are constructed to highlight particular causes, consequences, or moral lessons, often reflecting contemporary social, political, or cultural priorities \citep{entman_framing_1993}.

It assigns meaning, moral significance, and emotional tone to narratives, influencing how people interpret the past. In nationalist and populist discourse, framing often centres on themes of victimhood and injustice, portraying the nation as a heroic victim battling internal and external enemies, such as the European Union or liberal elites \citep{riedel_tri-marium_2022}. This way of framing limits the diversity of memories by rejecting alternative perspectives and labelling opposing groups as foreign, suspicious, or ideologically wrong \citep{piotrowski_between_2010}.
Memory is, therefore, interpretive and political by nature \citep{kucia_europeanization_2016}. Memory is not just a description of what took place, but a contentious field used to build identities, assert political projects, and make claims of moral authority in the present \citep{brown_making_1998}.

These narratives are not neutral because they serve national goals and help build identity. Other than framing, a narrative is a structured story or account of events. It has a beginning, middle, and end, often with characters, actions, and causality. A narrative explains what happened and may include interpretation or moral lessons \citep{bruner_narrative_1991}.

In Poland, the Law and Justice (PiS) party presents itself as the protector of national memory by focusing on stories of Polish suffering and heroic resistance, especially against communism and foreign control \citep{malksoo_memory_2009}. Narrative is thus a deliberate and active tool in the realm of memory politics \citep{wertsch_narrative_2008}. After Poland became a democracy in 1989, the economic and social transformation led many to feel disillusioned and excluded. Instead of perceiving the changes as a form of freedom, parts of the population saw them as a betrayal. This change gave them a strong sense of being victims, not against the external world, but because their leadership had fallen apart and because the promises of democracy had not been kept. PiS frames post-transition complaints as part of a broader story of national injustice \citep{piotrowski_between_2010}. Thus, the usage of memory is not just about external danger.

Furthermore, the politics of memory become even more intense through securitisation. Competing historical narratives frequently clash and are depicted as serious threats to national identity and unity. In this context, "memory laws" \citep{malksoo_memory_2015}, such as the one in Poland in 2018, serve as tools of security policy, aiming to maintain a narrow, state-approved version of the past while excluding those perceived as a threat \citep{zuk_anti-semitic_2023}. This securitisation instils a sense of uncertainty by eroding a clear national identity and replacing shared memories with polarising myths \citep{subotic_political_2018}.

In this context, narrative and framing shape memory into a political and ideological tool. Strategic memory building and securitisation strengthen dominant versions of history while supporting broader political goals, especially in times of populist rise and post-transition uncertainty.

\subsection{Memory is Power}
\label{sec:memory_is_power}

Nowadays, memory is increasingly transnational and multidirectional. Driven by globalisation and digital media, memories cross borders and are shared through global networks and cultural references, making them easily usable as a political tool \citep{assmann_transnational_2014}. Governments actively work to influence collective memory by controlling rituals of commemoration, public language, and memory work in institutions, thereby policing the boundaries of what should be remembered and what should be forgotten \citep{meijen_populist_2024}.

This is expressed in the public sphere, a significant space for power and influence. Collective memory is shaped through various spaces, political, social, and emotional, where it is not only passed down but also actively created through shared experiences across generations \citep{wydra_generations_2018}. Public memorials, museums, and rituals give these fleeting memories a lasting physical form, creating what scholars refer to as "memoryscapes" \citep{bresco_de_luna_end_2017} or "sites of memory" \citep{olick_social_1998}. For the political realm, this makes it interesting to control and legitimise dominant narratives. Because collective memory is highly subjective, gaps emerge that allow actors, "memory entrepreneurs", to challenge or rewrite these narratives.

These actors include academics, artists, activists, and grassroots groups who expose the fragility of official memory and highlight the political power of resistance \citep{langenbacher_twenty-first_2008}. Thus leading to closed memory communities \citep{assmann_transnational_2014}.

In post-Communist Eastern Europe, memory politics often follow a "politics of certainty", where the past is interpreted in rigid ways to strengthen national identity and political legitimacy \citep{dujisin_reassessing_2024}. These memory regimes are not fixed but rather change in response to shifting political goals, generational shifts, and new strategies \citep{bernhard_notitle_2014}. The past is frequently reshaped to support anti-Communist agendas, turning complex historical traumas into simplified nationalist stories \citep{grabowski_memory_2018}.

Institutionalisation of memory control is reflected in the concept of the "memory regime". A memory regime is an ordered system of elites, institutions, and symbolic resources that constructs, sustains, and imposes dominant historical narratives while actively repressing alternatives \citep{langenbacher_twenty-first_2008}.

As permeable as they are, national memory regimes retain considerable political influence. Memory remains the foundation of political culture, influencing institutional policy, legitimising governance, and supporting claims to national sovereignty \citep{subotic_political_2018}.

In the Polish case, the PiS government has leveraged state institutions and public symbols to anchor its preferred historical narrative, systematically marginalising more pluralistic or critical alternatives. Memory regimes take over through institutionalisation, legal conformity, and cultural transmission. They are implanted within the symbolic apparatus of the state and communicated through rituals of remembrance, school textbooks, and public debate. However, memory is always a site of contestation, where competing groups struggle to define and control the meaning of the past \citep{bernhard_notitle_2014}. This type of memory is always open to dispute, as seen in mnemonic narratives \citep{assmann_transnational_2014}.

In illiberal contexts, this often takes the form of defunding, securing, or erasing alternative histories, thereby strengthening a single, uniform national memory \citep{kapralski_jews_2017}. Illiberalism is a political approach that retains some democratic procedures, such as elections, but weakens the core principles of liberal democracy: limited government power, state neutrality, and an open, pluralistic society \citep{zakaria_rise_1997}. It often undermines constitutional checks, restricts rights, and favours majoritarian or traditionalist values over individual freedoms \citep{krastev_age_2011}. Illiberalism, therefore, includes a variety of ideological forms such as authoritarian, populist, and paternalist projects that share a common orientation against the liberal democratic order \citep{enyedi_concept_2024}, and one way these forms manifest is through the reshaping of memory, which undermines democratic discussion and hinders efforts for justice and reconciliation after political change. Instead of open debate, competing versions of history clash violently, making memory a source of division rather than unity. Elites are demonised, transitional periods are dismissed, and shared history is rewritten to serve narrow political goals \citep{bresco_de_luna_end_2017}. Memory becomes not a path to understanding, but a stage for power struggles, identity conflicts, and the exercise of control.

The transnational spread of memory challenges state efforts to control memory but also reflects that state-led memory politics remain extremely strong \citep{verovsek_caught_2021}. Memory serves as symbolic capital \citep{bourdieu_forms_1986}, a resource strategically mobilised in geopolitical and cultural arenas. This is one articulation of what has come to be referred to as "mnemonic capital", the strategic utilisation of victimhood and memory in securitised, competitive global contexts \citep{woycicka_mnemonic_2024}.

Past suffering is no longer confined to national remembrance but is utilised to secure moral status and global recognition. By silencing opposition and rewriting the cultural space, hegemonic actors impose a uniform understanding of the past, one that justifies exclusivist political narratives and quenches pluralism \citep{assmann_transnational_2014}.


\subsection{Mnemonic Warriors and Memory Laws}
\label{sec:mnemonic_warriors_and_memory_laws}

The concept of mnemonic war refers to the deliberate struggle over memory, in which political actors seek to control interpretations of the past to advance their present agendas. The central figure in this contest is the "mnemonic warrior": a political representative who enforces a single version of history while silencing alternative voices. Such figures construct official narratives that dictate what may be remembered and what must be forgotten, portraying any competing memory as false or harmful \citep{bernhard_notitle_2014}.

Mnemonic warriors operate in illiberal regimes, particularly in Central and Eastern Europe, to construct national identity and consolidate political power. They employ institutions and law to codify their version of history, with memory laws serving as their chief instrument. These laws do not protect historical truth but instead enforce a state-sanctioned narrative, embedding it within legal frameworks and institutions \citep{gliszczynska_grabias_memory_2014}. In post-Communist societies, this often takes the form of highlighting anti-Communist themes, reducing complex historical traumas into simplified nationalist narratives \citep{grabowski_memory_2018}.

Mnemonic warriors are not the only actors engaged in shaping collective memory. Pluralists accept competing interpretations, abnegators refuse to engage politically with memory, and prospectives emphasise visions of the future. However, the most militant actors remain mnemonic warriors, who rely on selective remembering and forgetting to maintain their authority \citep{bernhard_notitle_2014}. They erase the histories of minorities, reframe episodes of violence, and cast the nation as a perpetual victim incapable of wrongdoing \citep{woycicka_mnemonic_2024}. Populist regimes in particular are sustained by mnemonic warriors, who impose a mythologised, one-dimensional view of history and dismiss alternative accounts as distorted. Dissenters are labelled unpatriotic and pushed out of public life \citep{pozarlik_momentarily_2022}.

They would criminalise voicing opinions that go against the official version of history. Memory laws are legal tools that favour only one version of the past. These laws protect national honour, but at the same time, limit freedom. Furthermore, legal power is granted to one version of history, silencing other versions from being given voice \citep{grabowski_memory_2018}.
This is only part of a broader political movement. Illiberal governments use parliaments and courts to enact legislation, such as memory laws,  that appear democratic but is used to promote slim ideologies \citep{partner_what_2023}. Additionally, institutions, education, and symbols can be used by mnemonic warriors to become masters of memory and assert dominion over history \citep{bernhard_notitle_2014}. They present their version as the only just and moral one. They utilise it to create loyalty, stifle criticism, and shape identity \citep{bernhard_notitle_2014}. By doing this, they transform memory into a tool of discipline and power, rather than one for contemplation and public debate \citep{woycicka_mnemonic_2024}.

States shape collective memory through laws, cultural practices, and the regulation of rituals, narratives, and institutions, deciding what is remembered and what is forgotten \citep{meijen_populist_2024}. Although memory circulates globally and challenges national control, state-led memory politics remain highly influential \citep{verovsek_caught_2021}. Historical suffering, once tied mainly to national contexts, is now mobilised to gain moral authority and international recognition \citep{assmann_transnational_2014}. In this process, historical analogies are chosen for ideological purposes rather than objective reasoning, underscoring the need to reevaluate how history influences contemporary international politics and policymaking \citep{mumford_parallels_2015}.

\section{Populism}

The following section develops a framework to identify anti-pluralist tendencies within populist parties, examining the conditions that enable them, and tracing the causal mechanisms behind their varied outcomes. In contrast to much of the existing literature, this thesis rejects the assumption that populism is inherently anti-pluralist or undemocratic. Instead, populism is conceptualised as a mode of mobilisation against political elites who are perceived as failing to represent segments of the population \citep{urbinati_2019}. Building on this, populist parties are defined as those that endorse a division between "the pure people" and "the corrupt elite," and promote politics as the authentic expression of the "\textit{volonté générale}" \citep{rooduijn_populist_2024}.

The democratic character of populism remains a topic of widespread debate. Some scholars argue that populism is inherently anti-democratic because it frames politics as a moral struggle and delegitimises opponents, presenting populists as the sole authentic representatives of "the real people" \citep{muller_what_2016}. This view defines anti-pluralism and elite delegitimisation as central features of populism \citep{pozarlik_momentarily_2022}. In contrast, others suggest that populism is not necessarily anti-democratic, and can act either as a threat or a corrective, depending on its form and context \citep{mudde_populism_2012}. Other scholars emphasise populism as a mode of mobilisation against elites who are seen as failing to represent significant segments of society \citep{urbinati_2019}.

Populism is broadly understood as a "thin-centred ideology" that encompasses politics in a fundamental antagonism between two morally opposed sides: "the pure people" and "the corrupt elite" \citep{couperus_memory_2023}. Although definitions of populism vary, the concept of "thin populism" is widely accepted \citep{mudde_populism_2017}. This framework conceptualises society as divided into two homogeneous groups, the people and the elite, assumes antagonism between them, appeals to popular sovereignty, depreciates the elite, and exalts the people \citep{stanley_thin_2008}.

It can be defined more as an elastic political rationality rather than a complete doctrine. A simplified binary worldview, anti-elitist rhetoric, and affective mobilisation around crises, betrayal, and national renewal mark populism. Populist rhetoric often leverages fears of the "other" and social transformation, using history to disqualify rivals and redefine cultural narratives \citep{zuk_anti-semitic_2023}. When in power, populist leaders are likely to reconfigure the state to represent these concepts, making it an instrument of cultural resistance and nationalist rejuvenation. Even if populism can manifest at any point along the political spectrum, right-wing populist forms are particularly susceptible to xenophobia, historical revisionism, and exclusive nationalism \citep{forchtner_trajectory_2019}. Conceptually, at its core, populism places "the people" as a real, morally good constituency under threat by strong, often foreign-allied elites.

Right-wing populist parties often instrumentalise ethnic, religious, linguistic, or political minorities as scapegoats for societal problems, portraying these groups as dangerous threats "to us" or "our" nation \citep{wodak_politics_2015}. This strategy manifests as a "politics of fear" and is frequently accompanied by what can be described as an "arrogance of ignorance," where appeals to common sense and anti-intellectualism mark a return to pre-modernist rational thinking \citep{wodak__2017}.

In this context, right-wing populism can be defined as a political ideology that challenges existing political consensus and frequently combines laissez-faire liberalism with anti-elitism. Law and Justice (PiS) politics in Poland is often analysed through the lens of populism. Liberal and conservative journalists frequently critique PiS' social redistribution schemes, such as the 'Family 500+' programme, while academic literature emphasises that the party employs classic populist representations of society and politics \citep{cadier_populism_2020}. PiS exhibits a populist character through its moralistic rhetoric, which pits ordinary citizens against corrupt elites. However, as \citet{bill_counter-elite_2022} argues, PiS's populism does not target elites in general but specific "liberal" and "post-communist" elites whom the party seeks to replace with its own supporters. Similarly, \citet{ost_workers_2018} notes that PiS positions itself as a defender of disadvantaged citizens, reinforcing its populist appeal.

This reduction of complex political issues to binaries of right vs. wrong  that make policies appear to embody "the will of the people" \citep{couperus_memory_2023}. This divided moral world allows populist leaders to simplify complex political or historical issues into emotional stories about loss, unfairness, and hope for renewal. Populism, therefore, is not so much concerned with the content of policies, but with style and strategy \citep{forchtner_trajectory_2019}. Some authors have described populism as a "political style" that captures an elusive political phenomenon \citep{moffitt_global_2016}.

Right-populist discourse is founded on crises and high levels of emotion, such as fear and resentment, and create a condition of urgency \citep{forchtner_trajectory_2019}. Their actors do not merely talk about the past; they remake it, using memory as a political tool. This reinforces social cleavages and politicises historical awareness. Populist memory affirms national myths of grandeur, heroism, and victimhood \citep{subotic_political_2018}. These myths normalise identity politics and make illiberal policy more palatable, constantly  placing the nation in the proper and heroic, or wronged, and tolerating a limited scope for nuance or shared responsibility. Populists in post-communist Central and Eastern Europe employ memory politics to appeal to national victimhood at the hands of past regimes, framing criticism as victimisation \citep{woycicka_mnemonic_2024}. Recognition of local Holocaust involvement, for example, is often interpreted as an attack on national honour \citep{grabowski_memory_2018}. Populists construct narratives of collective past traumas to establish a shared identity and mark symbolic divides between "us" and "them" \citep{wydra_generations_2018}.

\subsection{Othering}

The "us versus them" reasoning is not a secondary feature but the organisational norm of populist memory politics. Through the construction of external enemies and domestic traitors, populist forces consolidate national identity, obliterate critical memory, and rende illberal rule legitimate. Othering is a mnemonic governance device that reworks past accounts to legitimise present power \citep{subotic_political_2018}.

Nationalism is also grounded in the construction of clear boundaries between "us" and "them," and populism utilises memory politics to reinforce these divisions \citep{bieber_is_2018}. By selectively remembering certain events and forgetting facts that are inconvenient to the narrative, populists portray the nation as heroic and innocent, attributing past atrocities to external agents such as the Soviets or Nazis while erasing local complicity \citep{grabowski_memory_2018}. Through the concept of "banal nationalism," national belonging is reinforced in daily life via media, including news and social media posts. National identity thus becomes a primary, emotionally powerful identity, especially when early memories are shaped by repeated social cues, such as media narratives \citep{jamieson_theorising_2002}.

Nationalism functions to demarcate "us" from "them," prioritising group identity over others. Political and media elites exploit this to mobilise populist support, disseminating hard, exclusionary nationalist discourse \citep{bieber_is_2018}. Populists leverage narratives of betrayal to form affective bonds with "the people," framing history as a moral struggle between national truth and elite deception, and asserting moral authority over the nation's past \citep{woycicka_mnemonic_2024}. This mnemonic politics drives public mobilisation through media, events, and legislation that regulates collective memory, effectively suppressing critical or plural perspectives \citep{jamieson_theorising_2002}.

Stories of national grievance are not always about what happened but are instead actively used to build enemies and to legitimate exclusion \citep{forchtner_trajectory_2019}. The "other" can be depicted by groups described as challenging traditional values and manipulating national memory. In nationalist rhetoric, such cultural and demographic "others" are often racialised or used as an existential threat to the nation \citep{subotic_political_2018}.

Finally, othering is not only a rhetorical tactic of populist discourse but also a necessary process through which historical memory becomes weaponised to construct identity, exclude opposition, and attain political authority \citep{subotic_political_2018}. Memory is weaponised as a political instrument that is used to glorify victimhood, abolish collective guilt, and assign membership. Populists in this context are mnemonic warriors, using simplistic, myth-encrusted histories to rally mobilisation and legitimise their rule \citep{bernhard_notitle_2014}.

Othering is a fundamental mnemonic practice that functions to reinforce a unified national identity by outsourcing responsibility and marginalising oppositional histories \citep{subotic_political_2018}. The populist collective memory is constructed on the post-communist Central and Eastern European "us versus them" logic of a binary. Geopolitical accounts share a logic in which Russia is portrayed as a permanent aggressor, Germany as a moralising outsider \citep{riedel_tri-marium_2022}, and the EU as a liberal empire determined to erode national sovereignty \citep{verovsek_caught_2021}. Parties like PiS position themselves as defenders of national identity against such actors \citep{subotic_political_2018}.

\subsection{Conspiratorial Character}

Populism often overlaps with conspiratorial thinking. Leaders use conspiracy theories to garner support, discredit their opponents, and portray politics as a struggle between a virtuous people and corrupt elites. In this framing, the populist leader becomes the nation's sole defender \citep{mudde_populism_2017}. Building on the previous discussion of othering, this section examines how conspiratorial populism intersects with antisemitism, situating the phenomenon within the Eastern European context, particularly in Poland.

Antisemitism as an ideology is not only a persistent form of hostility toward Jews, but also a worldview. Antisemitism is mainly expressed through negative attitudes, discriminatory actions, or cultural and political representations towards Jews \citep{postone_anti-semitism_1980}. Common stereotypes depict Jews as secretive, manipulative, or responsible for economic, political, or social crises. The International Holocaust Remembrance Alliance similarly defines antisemitism as a "certain perception of Jews," expressed in speech, writing, visuals, or actions, often blaming Jews for societal misfortune and portraying them as conspiratorial actors. Antisemitism frequently conflates Jewish identity with conspiratorial control over global or national systems, attributing complex social problems to a malevolent group \citep{ihra_ihra-non-legally-binding-working-definition--antisemitism-1_2016}.

As previously mentioned, antisemitism is not only a form of discrimination, but also an ideology and, much broader, a worldview. Antisemitism is an act of vigour. In what \citet{adorno_elements_2002} call pathic projection, the antisemite projects anything and everything onto Jewish people, whom he sees as "the elites" \citep{adorno_elements_2002}. Through this act of pathic projection, there is not even a need for actual Jewish people for the antisemite to live out his passion. Thus, any conspiracies will eventually lead to Antisemitism by virtue of projecting internal fears onto a symbolic "other" \citep{rensmann_politics_2017}.

\citet{postone_anti-semitism_1980} distinguishes antisemitism from racism by noting its tendency to project abstract processes, such as capitalism or global finance, onto a personalised, conspiratorial enemy. In this sense, antisemitism functions as a social logic: it transforms impersonal domination into the intentional actions of a hidden, identifiable group \citep{postone_anti-semitism_1980}. Antisemitic conspiracy theories have historically proven to be durable \citep{demata_anti-sorosism_2022}. Such narratives adapt to new historical contexts, demonstrating the flexibility of antisemitism as a worldview.

\citet{mudde_populist_2007} identifies nativism as the core ideological feature of populist radical right parties in Europe. He defines it as the belief that states should be inhabited exclusively by members of the native group. At the same time, non-native persons and ideas are perceived as a fundamental threat to the homogenous nation-state. Nativism operates through a process of othering, in which the nation is constructed by demarcating a sharp boundary between those considered native insiders and those marked as dangerous outsiders. Antisemitism represents one of the most enduring historical manifestations of this logic \citep{mudde_populist_2007}.

Populist discourse often presents crises, economic, political, or cultural, as the result of manipulation by a hidden, malevolent elite. The elite is depicted as shadowy, cosmopolitan, and manipulative, echoing antisemitic codes even when Jews are not explicitly mentioned \citep{wodak_politics_2021}. Historical antisemitic texts, such as The Protocols of the Elders of Zion, exemplify systematic conspiracies, portraying a single group orchestrating wars, revolutions, and economic crises \citep{simonsen_antisemitism_2020}. Event-specific conspiracies, such as the alleged perpetrators of 9/11 or the JFK assassination, function as episodic manifestations of broader systematic conspiracy narratives \citemultp{demata_anti-sorosism_2022,wodak__2017}.

Jews were long portrayed as the ultimate "internal outsiders", present within European societies yet imagined as alien, subversive, and incapable of belonging to the national community. Mudde's typology of enemies explicitly includes Jews, alongside Muslims and Roma, as recurring figures cast as threatening non-natives within radical right discourse \citep{mudde_populist_2007}. Even in contemporary contexts where overt antisemitism is less prominent, its patterns persist in rearticulated forms, such as conspiracy theories attributing globalisation or mass immigration to Jewish influence, often personified in the figure of George Soros \citep{demata_anti-sorosism_2022}.

In Eastern Europe, antisemitism has deep historical roots, shaped by religious and racialised prejudice. Stereotypes such as the "Jewish capitalist," the "intellectual Jew," the "Jewish Bolshevik," and the "anti-national Jew" depict Jews as conspiratorial actors manipulating political, cultural, and economic life \citemultp{demata_anti-sorosism_2022,wodak__2017}. Unlike Western Europe, where antisemitism often appears coded or implicit, Eastern Europe demonstrates explicit and historically grounded forms shaped by complex social, political, and wartime experiences \citep{wodak__2017}.

Thus, Bergmann's understanding of antisemitism seems fitting, since he situates it within a deeper history of what he terms national antisemitism. He explains that Jews were depicted not only as foreigners but also as those who failed or refused to assimilate into the national order, embodying a "national non-identity". This perception reinforced their construction as an irreconcilable other within the nation. Within this framework, Jews are portrayed as powerful and threatening, accused of betrayal in politics, of manipulation in finance, of exploiting Holocaust memory to discredit nations or claim illegitimate compensation, and of undermining Christianity and national culture through universalist values. Bergmann refers to this constellation as "classical" or political antisemitism, a complex of ideas that persists in conspiracy theories about Jews as a hidden collective power \citep{bergmann_antisemitism_2013}.

Unlike much of Western Europe, where Reformation and Enlightenment currents undermined medieval myths, Polish Catholic clergy continued to perpetuate accusations such as ritual murder, host desecration, and the charge of Jews as "Christ killers" well into the eighteenth century. At the same time, the close coexistence of Jews and Christians in Poland produced ambivalent social dynamics. While daily contact sometimes generated cooperation and even intimacy, Catholic preachers used polemical narratives to police religious boundaries, presenting Jews as dangerous outsiders and spiritual enemies of Catholic Poland \citep{teter_jews_2005}. This historical continuity shows how antisemitism, while adapting to new contexts, remains embedded in the broader nativist logic that underpins exclusionary politics in Europe and continues to shape political and cultural debates in Catholic countries such as Poland \citep{bergmann_antisemitism_2013}.

In the Polish context, tensions around the competition of victimhood between Poles and Jews over Nazi crimes have influenced antisemitic discourse \citep{bergmann_antisemitism_2013}. Poland exemplifies contemporary intersections of populism and antisemitism. Since 2015, the United Right coalition, led by Law and Justice (PiS), has implemented the "good change" program, restructuring institutions to replace a perceived "bad" elite with a "good" one that represents the nation's interests \citep{bill_counter-elite_2022}. Alongside these reforms, antisemitism resurfaced in political and media discourse. In early 2018, tensions arose over historical memory debates, including legislation addressing the phrase "Polish death camps," linking national honour to Holocaust narratives and illustrating the interplay of populist politics, nationalist sentiment, and historical regulation \citep{pankowski_resurgence_2018}.

In Poland and Eastern Europe, these narratives serve not only political ends but also in the formation of collective identity, defining both the "Other" and the internal elite as threats. Populist leaders cultivate moral urgency by portraying society as besieged by external enemies and internal traitors. Conspiratorial thinking thus intersects with historical antisemitic narratives, legitimising authority while perpetuating social division and exclusion \citep{bergmann_strategic_2025}.  Overall, antisemitism exemplifies how conspiratorial populism operates: it interprets crises through conspiratorial frameworks, delegitimises elites, and positions leaders as defenders of the people. In Eastern Europe, where historical memory and national identity remain highly contested, antisemitic tropes provide fertile ground for populist narratives. By framing society as threatened by malevolent forces, conspiratorial populism fosters collective vulnerability, setting the stage for the subsequent discussion on victimhood as a central mobilising strategy.

\subsection{Victimhood}

Victimhood has emerged as an overbearing trope in collective identity and political legitimacy, particularly in post-communist Central and Eastern Europe. It is not just an account of past pain, but also a performative and mobilising one that political actors strategically use. Victimhood populism is therefore understood as the strategic use of collective trauma to mobilise the nation against so-called internal and external threats \citep{meijen_populist_2024}.

Memory politics creates a moral framework in which the nation is seen only as a hero or a victim, never as an aggressor \citep{brants_transitional_2013}. The nation is portrayed as always in danger, always healing, and always anticipating new betrayals, fostering emotions that encourage authoritarian rule in the name of patriotic resilience \citep{bernhard_notitle_2014}. Polish victimhood is both a political strategy of narrative and not a factual recollection. As a result of its employment within populist politics, victimhood is utilised as mnemonic governance and a national mythological foundation. Victimhood simplifies history into a moral tale, silences critiques, and provides leeway for illiberalism in the present, all in the name of moral reeducation and shared memory \citep{meijen_populist_2024}.

One of the most essential features of victimhood populism is its selectivity. Polish hardships as victims of Nazi and Soviet occupation are emphasised. Simultaneously, more contentious elements of the past, such as collaboration with the Holocaust, anti-Semitic incidents, or ethnic cleansing operations, are repressed or reinterpreted \citep{stanczyk_commemorating_2014}. Selective memory reconstitutes victims, remaps perpetrators, and silences inopportune facts \citep{grinchenko_introduction_2018}. Victimhood in such contexts is not a spontaneous act of recollection but a politically powerful tool. Its power over memory comes from its ability to unite the nation, raise its moral status, and justify exclusionary or strict policies by claiming to defend historical justice \citep{forchtner_trajectory_2019}.

\subsection{Role of the Media}

In theoretical terms of memory politics, the media have central functions in forming, circulating, and contesting collective memory. Digital technologies, particularly social media, have transformed the way collective memory is formed \citep{assmann_transnational_2014}. Whereas traditional media once offered centralised and fixed forms of narrative, digital media promotes a more disseminated, affective, and dialogical mode of remembering \citep{woycicka_mnemonic_2024}.

Drawing back to the definition of populism in this context, scholars argue that populism should be analysed as a performative style, emphasising its expression in the public sphere \citep{moffitt_global_2016}. The public sphere is a communication system that mediates between organised and informal deliberations across different political levels \citep{habermas_political_2006}. The structure of national media shapes the public sphere and influences the visibility and framing of populist discourse. Media both shape and reflect political culture, playing a decisive role in legitimising or delegitimising anti-pluralist claims. Media can respond to populist parties in various ways: by ignoring them, exposing them, or normalising their agenda through repeated coverage \citep{crum_renewing_2020}.

In addition to state media, right-wing political forces in Central and Eastern Europe also rely on informal digital networks to reframe contentious histories. Their narratives crossed borders, intertwining each other's tales of grievance and moral exceptionalism. Within this theoretical lens, media, in the form of digital media,  must be conceived as key instruments within populist memory regimes. They facilitate the emotional, selective, and strategic invocation of history, transforming collective memory into an arena of ideological struggle which reinforces exclusionary identities and illiberal rule \citep{malksoo_memory_2009}.

Populist and illiberal actors efficiently use online media to promote oversimplified, emotive histories about the past towards their political ends. Although such sites enable greater participation, they also facilitate the easier manipulation of memory and the dissemination of hegemonic, exclusionary narratives. In today's "media democracy", where spectacle, emotion, and scandal usually overwhelm facts, sites like X enable them to speak directly to their followers. This allows actors to bypass traditional media and spread emotionally manipulative, simplifications of historical messages that serve to reinforce their political agenda. Thus, the internet space becomes both a site of remembrance and a site of struggle over the interpretation of history \citep{forchtner_trajectory_2019}.

Populist actors engage in mnemonic warfare, invoking history as a political tool to determine national identity. Their narratives focus on national suffering and betrayal, often disseminated via memes, brief clips, and online indignation timed to coincide with anniversary commemorations or heightened political tensions \citep{bernhard_notitle_2014}.

Virtual media expand the affective and performative spaces of the narratives. Likes, shares, and hashtags make it easy to spread simplified hero stories or victim stories, reifying "us vs them" identities and deepening social cleavages \citep{simko_marking_2020}.  Digital technologies have created a transnational space of memory that transcends institutional frameworks and national boundaries. Social media transnationalises and makes memory participatory, allowing anyone to rewrite history \citep{assmann_transnational_2014}. However, social media algorithms favour emotive, divisive, and conflict-driven content that can reinforce fragmentation and fuel memory conflict \citep{woycicka_mnemonic_2024}.

Populist governments also use their media influence to promote selective, heroic accounts of national history and hide uncomfortable truths, such as past collaboration or persecution of minorities. In this way, the media act as a soft authoritarian apparatus embedded in everyday life \citep{bernhard_notitle_2014}.

This chapter established the theoretical framework for explaining how collective memory is politicised and contested as a resource in illiberal and populist contexts. Populism, particularly right-wing populism, is not only a political style but also a moralised vision concerned with simplifying complex histories into oppositions of good and evil, portraying the nation as virtuous and innocent, and its enemies as corrupt outsiders \citep{mazzini_three-dimensional_2018}. \pagebreak This process of othering is the basis for populist narratives and tends to foster exclusionary national identity \citep{bieber_is_2018}. This framework provides the context for examining the Polish case in its entirety, specifically how memory is constructed, mediated, and weaponised through institutionalised platforms and social media within the context of election contests.
