\chapter[Methodology]{Methodology}
\label{Chap:Methodology}

This chapter outlines the research methods and approaches employed to examine the construction of political memory and identity by the Law and Justice party (PiS) in Poland during the 2015 and 2023 election campaigns. The study combines Critical Discourse Analysis (CDA) and Digital Ethnography to explore both the textual and digital dimensions of mnemonic narratives. Through these complementary methods, the research investigates how memory is strategically shaped, communicated, and contested across traditional and social media platforms. The chapter also details the data collection strategies, including the use of the 4CAT toolkit for gathering social media data and the use of LLMs for translation. It discusses the ethical considerations and challenges of working with digital content in a rapidly evolving media landscape.

\section{Methods Selection}

The research analyses news coverage from various Polish media outlets with different political orientations. It also entails a comparison of the Law and Justice Party's (PiS) election manifestos of 2015 and 2023, to trace developments in ideology over time. The thesis applies Critical Discourse Analysis (CDA) as its core methodology, drawing upon Foucault's notion that discourse is a means of producing knowledge in a society \citep{tracy_critical_2015}.  Following Foucault, this perspective posits that discourse shapes and controls how we understand people, activities, and concepts, dictating what is considered correct knowledge \citep{wodak_critical_2013}. CDA in this research is applied to disassemble the discursive strategies employed by PiS in constructing political identities and collective memory \citep{weiss_critical_2003}.

The employed variation of CDA in this thesis is the Discursive Historical Approach (DHA). DHA conceptualises discourse as context-dependent semiotic practices that are socially constituted and socially constitutive, related to macro topics, and oriented toward argumentation \citep{tracy_critical_2015}. It focuses on ways of portraying oneself positively and others negatively, using techniques such as naming, describing, and framing, as well as language tools like metaphors, quotations, and references \citep{wodak__2017}. By combining social, political, and historical contexts, DHA systematically explores how discourses, genres, and texts evolve, distinguishing discourse as structured knowledge patterns and texts as unique realisations \citep{reisigl_discourse_2001}.

Six discursive strategies,  nomination, predication, argumentation, perspectivization, mitigation, intensification, and presupposition,  can guide the analysis, encompassing linguistic, visual, and multimodal practices aimed at achieving social, political, or semiotic goals \citep{jackson_john_2019}.

The theoretical foundations of CDA are rooted in the insight that discourse is socially constitutive, context-dependent, and involved in broader macro-level subject matters and debates on truth and normative validity \citep{reisigl_discourse_2001}. It analyses how language functions within arrangements of domination and inequality, and how it reflects, constructs, and maintains power relations \citep{fairclough_discourse_1992}. Following van Dijk's inclusive perspective on Critical Discourse Studies, this research understands media narratives as central instruments of ideological reproduction and political control \citep{van_dijk_principles_1993}. Thus, CDS is understood as the framework of the research. CDS is defined as a critical, sociopolitical stance that is problem-oriented and committed to social justice and equality. Its primary goals are to analyse how discourse contributes to domination and inequality, challenge discursive injustice regarding human rights, and amplify the voices of marginalised groups while remaining accessible and not overly technical \citep{tannen_critical_2015}.

Digital Ethnography, as envisioned in this case, extends ethnographic practice into the online world, where political communication is increasingly taking place. It examines how individuals produce, negotiate, and exchange meaning in digital environments shaped by evolving data transmission technologies. There has been a proliferation of methodological approaches to online ethnography in recent years, sharing a common point: an emphasis on real-time social interaction in digital worlds \citep{delli_paoli_digital_2021}.

To process and group posts systematically, tools such as 4CAT \citep{peeters_4cat_2022} (deployed in Docker) and Zeeschuimer \citep{stijn_peeters_zeeschuimer_2025} were employed. For translation, an AI model, gpt-oss-20b \citep{openai_oai_gpt-oss_model_card_2025}, deployed with LM Studio, was used.

\pagebreak

\section{Technical Infrastructure}

To collect social media posts from Law and Justice (PiS) politicians during the final months of the 2015 and 2023 Polish parliamentary elections, this study used the browser plugin Zeeschuimer \citep{stijn_peeters_zeeschuimer_2025}. Zeeschuimer was used to collect posts semi-automatically and then upload the raw data into 4CAT. 4CAT is an open-source tool developed by the University of Amsterdam's Open Intelligence Lab. It enables researchers to gather and explore data from platforms such as Twitter, Reddit, and Telegram \citep{peeters_4cat_2022}.

4CAT offers a user-friendly web interface that does not require extensive programming knowledge and supports the analysis of text, images, videos, and audio. Integrating Zeeschuimer with 4CAT facilitated efficient capture, organisation, and preliminary processing of large datasets \citep{stijn_peeters_zeeschuimer_2025}. It supports transparent and reproducible research by preserving key data, such as timestamps, usernames, and post content. Only publicly available content was collected, which includes verified PiS accounts, politicians, and party-linked media. Data was archived through a mix of manual and semi-automated methods. The dataset used in this thesis was published on Zenodo to ensure transparency and reproducibility. All data discussed hereafter refer to this dataset and will be cited accordingly as \citep{rybicki_2025_16933320}.

According to X's User Agreement, users retain ownership of their posts but grant the platform a non-exclusive, worldwide, royalty-free license to use and share them. Public posts can be collected and used for academic research if researchers respect user privacy, adhere to platform rules, and refrain from misusing the content. Data should be retained only as long as it remains publicly available or anonymised when necessary \citep{noauthor_twitter_2023}.

Due to restrictions in X's public API and the requirement for a secure, self-hosted setup, 4CAT was deployed on a private server using Docker. Docker can be understood as a platform that enables developers to package software and its dependencies into standardised, portable environments, ensuring reproducibility across different systems. This approach enables the consistent deployment of complex tools without requiring users to configure each component manually or be hindered by variations on the host system. Docker enables software, bundled into so-called containers, to run consistently and portably across different systems. This containerisation method also allows research software to run in a standardised, portable environment \citep{boettiger_2015}.

\pagebreak 

For translating the collected posts, a type of Artificial Intelligence (AI) was used, specifically a Large Language Model (LLM). The type of LLM applied in this study belongs to the family of Generative Pretrained Transformers (GPTs), a model architecture designed for natural language processing tasks. GPT models can be broadly understood as highly advanced forms of predictive text generation: they are trained on extensive datasets, often amounting to petabytes of textual material, and learn to generate language by predicting the most likely continuation of a sequence based on patterns identified in the training data \citep{bubeck_sparks_2023}. The model used was gpt-oss-20b, which offers open availability, transparent documentation, and reproducibility features that make it particularly suitable for academic research \citep{openai_oai_gpt-oss_model_card_2025}. 

This integrated methodology allowed for the efficient collection, processing, and analysis of social media content, combining manual capture with automated semantic analysis. The workflow enabled sorting, filtering, and language analysis without the need for a predefined codebook, improving both flexibility and transparency in the research process.

\section{Research Approach}

The research adopts an inductive approach. The inductive method involves identifying familiar narratives, tropes, and framing devices related to collective memory and examining these over election cycles, sharpened by memory politics theory, such as the notion of mnemonic warriors, to untangle discursive continuities and breaks \citep{day_finding_2019}. This thesis unites the disciplines of Populism, Political Sociology, and Memory Studies. It examines the mnemonic strategies employed by the Law and Justice Party (PiS) in two pivotal election campaigns, 2015 and 2023, which both trace their roots to Poland's shifting political and historical environment.

The 2015 campaign marked PiS's comeback and was characterised by a national-conservative turn, as well as the revival of the memory conflict. By contrast, the 2023 campaign followed eight years of PiS rule, in the context of massive geopolitical and social changes, including EU tensions, the COVID-19 crisis, and the war in Ukraine. The two campaigns present an interesting empirical puzzle, in the form of a temporal difference, through which the changing dynamics of memory politics in Poland can be examined \citep{day_finding_2019}.

The blended use of Critical Discourse Analysis and Digital Ethnography allows for an integrated investigation of both the production and consumption of memory narratives. \pagebreak Both accounts address the issue of how political actors craft mnemonic discourses and subsequently engage with, revise, or respond to digital publics \citep{matthes_digital_2017}.

\section{Rationale for Method Selection}

This research takes a mixed-method, interpretivist approach, focusing on how meaning and narratives are constructed. This choice is based on the understanding that modern political communication, especially populist rhetoric, is not only shaped by ideology but also plays a significant role in shaping collective memory. In today's social media era, the once-clear hierarchy of collective memory has broken down \citep{rutten_memory_2014}. While traditional mass media once played a dominant role in framing national narratives and defining historical authority, digital platforms now enable competing interpretations of the past to proliferate \citep{wasilewski_radical-right_2023}.

Critical Discourse Studies (CDS) examines how ideologies are discursively constructed and reproduced, even though they are not directly observable in texts. Ideologies are underlying belief systems that shape group opinions and legitimise dominance. They operate through layers such as group attitudes, event models, and context models. They may be analysed using schemas that include group membership, typical acts, goals, social relationships, and access to discourse and resources \citep{van_dijk_principles_1993}. CDS treats discourse not merely as language, but as a tool for both social control and resistance, recognising the mutually constitutive relationship between discourse and society. Discourse is understood in CDS as multidimensional, encompassing linguistic objects, forms of action, modes of interaction, social and cultural practices, mental constructs, and economic commodities. Power and dominance are exercised and contested through discourse via mechanisms such as control over speech acts, topic selection, lexical choices, and rhetorical strategies. CDS frames scholarly work as interventionist, aiming to uncover discursive injustices and contribute to social change, thereby emphasising the analytical and normative dimensions of discourse research \citep{tannen_critical_2015}.

 This study employs the Discourse-Historical Approach within CDA  to facilitate an interdisciplinary, context-sensitive examination of discourse as a socially and historically situated practice  \linebreak\citep{reisigl_discourse_2001}. CDA provides tools to analyse how language and other semiotic resources construct political identities, historical narratives, and ideological boundaries, while considering the interaction of text, cognition, and societal structures. The interdisciplinary nature of CDA has established it as a robust analytical instrument across various disciplines, including sociology, history, media studies, and political science \citep{weiss_critical_2003}.

 Critical Discourse Analysis, following the DHA, is used to analyse political speeches, campaign materials, and official online posts. This method facilitates an understanding of how historical narratives, identity markers, and ideological positions are linguistically constructed and contextually grounded. The DHA enables the tracing of discursive strategies such as nomination, predication, and argumentation, allowing for the identification of recurring mnemonic tropes and recontextualisations across campaign periods \citep{reisigl_discourse_2001}.

To complement this discourse-centric approach, the study incorporates Digital Ethnography to investigate how these mnemonic narratives are shaped, circulated, and interpreted in online spaces. This method is particularly suitable given the shift from mass to digital media in the contemporary media landscape. Digital ethnography enables a detailed examination of user engagement on platforms such as Twitter and news websites. Rooted in classical ethnography, digital ethnography adapts its tools to the online realm while retaining its commitment to exploring lived experiences and meaning-making \citep{matthes_digital_2017}.

The increasing role of digital platforms in shaping memory has led scholars to describe this transformation as the emergence of "technologies of memory". These technologies, including smartphones, social media, and digital archives, extend human memory beyond the limitations of internal cognitive processes. The internet plays a unique role by offering vast, rapidly changing, and easily accessible information, which transforms how memory is stored and retrieved. While the implications of this shift are debated, scholars agree that it necessitates a more nuanced understanding of memory-making in the digital age \citep{van_house_technologies_2008}.

The rise of digital communication has led to an unprecedented volume, speed, and variety of data production, driven by mobile technologies, low-cost storage, and widespread internet access. While this data explosion enhances the richness of available material for political and cultural analysis, it also presents significant challenges for selecting meaningful data, long-term preservation, and ethical use. Digital memory is often fragmented, ephemeral, and susceptible to loss due to platform-specific limitations, link rot, or changing privacy settings. This raises questions not only about what can be collected and preserved, but also about whose memories are prioritised or marginalised in the digital public sphere \citep{van_house_technologies_2008}.

\citet{menke_digital_2023} focus on the interplay between digital memory and populism. They highlight how populist actors increasingly leverage digital platforms to shape collective and personal memories, constructing nostalgic narratives of a "heartland" past and mobilising support against perceived elites. Digital memory enables populists to circumvent traditional media, integrate their narratives into everyday life, and shape societal discourse. Thus, emphasising the role of memory in political discourse online \citep{menke_digital_2023}.

Populist communication has not only reshaped national and global politics but also penetrated everyday interactions and people's relationships with them. The pervasive nature of populism complicates its definition as a single phenomenon, as it can manifest as a political style, strategy, thin ideology, pathology, or form of discourse \citep{menke_digital_2023}.

Despite these technological changes, ethnography remains fundamentally about telling social stories and capturing lived experiences. New digital tools, such as social media platforms, expand the possibilities for data collection, allowing for greater detail, multimodality, and interaction. At the same time, these developments raise critical ethical questions around privacy, informed consent, and the visibility of research subjects \citep{murthy_digital_2008}. In this decentralised mnemonic landscape, media do not merely mediate memory; they constitute a primary terrain in which it is contested, negotiated, and reshaped \citep{wasilewski_radical-right_2023}.

Large Language Models, such as Generative Pretrained Transformers, are advanced neural networks trained on vast amounts of text data. They operate by converting text into embeddings, numerical vector representations of word meaning, enabling them to capture semantic relationships between words. This makes LLMs particularly effective for translation tasks, as words across different languages may not share identical vectors, yet their proximity within the embedding space allows for accurate transfer of content and context between languages on which the model has been trained. Unlike systems that rely on memorisation, LLMs learn statistical patterns in language, allowing them to generate coherent new text. In this sense, they function similarly to an individual with extensive exposure to written language who, by relying on accumulated linguistic knowledge, can probabilistically anticipate the continuation of a phrase based on previously encountered patterns \citep{mansour2025language}.

The selection of gpt-oss-20b for this research is guided primarily by methodological considerations related to efficiency, reasoning capability, reproducibility, and flexibility. This local inference capability enables experiments to be conducted interactively without relying on cloud infrastructure, which is crucial for reproducible research and data privacy. From a reasoning perspective, gpt-oss-20b has demonstrated strong performance on benchmark tasks, supporting chain-of-thought prompting for structured outputs and enabling nuanced analysis of complex textual data. This capability makes it particularly suitable for research tasks that require multi-step reasoning, classification, and translation. Additionally, the model is openly licensed and fully documented, with accessible weights and model cards. This transparency aligns with open science principles, allowing other researchers to replicate and build upon the study. The model also supports long-context processing (up to 128k tokens; 1 token $\approx$ 4 characters of English text) and configurable reasoning depth, allowing researchers to balance speed and analytical depth according to task requirements \citep{openai_oai_gpt-oss_model_card_2025}.

\section{Data Collection Strategy}

This thesis examines the mnemonic strategies employed by the Law and Justice Party (PiS) across traditional and digital media during its 2015 and 2023 electoral campaigns. The primary data sources include party manifestos, news articles, and digital content disseminated via X (formerly Twitter). The academic literature and all secondary data are available in English, reflecting the broader academic landscape of English-language scholarship. Primary data sources are in Polish. Although I am fluent in both English and Polish, I used DeepL and gpt-oss-20b as translation tools to assist with rendering Polish texts into English, particularly for direct quotations.

For the systematic analysis of social media content, a hybrid approach combining manual data collection with automated processing was employed. Initially, relevant content from platforms such as X was captured using the browser extension Zeeschuimer. This tool enables researchers to collect and store information about the elements displayed while browsing social media platforms, including TikTok, Instagram, X/Twitter, LinkedIn, and others. Collected data was directly uploaded into a 4CAT instance for structured storage and further analysis \citep{stijn_peeters_zeeschuimer_2025}.

Data collection was conducted through manual scrolling, with all relevant content systematically archived following the retrieval process. For pre-processing, the collected material was subsequently organised and stored within the 4CAT infrastructure. Two comprehensive datasets were then constructed: one comprising all accounts examined in 2015 and another containing those from 2023, generated by applying account and date filters.


\section{Analytical Procedures}

The foundation of this research is a comparative inquiry into the evolution and continuity of mnemonic narratives employed by the Law and Justice Party (PiS) across two critical electoral campaigns. This comparison is not incidental but stems from a central empirical puzzle: how and why the political memory strategies deployed by the exact actor change, or persist, over time. This thesis compares PiS's election campaigns to understand how its memory narratives evolve or remain consistent over time. By examining these various moments, the research demonstrates how political memory adapts to shifting social and political conditions while retaining its core ideas. This time-based comparison helps answer what parts of the narrative persist, what changes have occurred, and why. It offers insight into how collective memory is shaped and reshaped in today's digital and media-driven political environment. \citep{day_finding_2019}.

Critical Discourse Studies is based on the Discourse–Cognition–Society framework, which considers discourse as the textual or spoken content with its linguistic, rhetorical, and stylistic features, cognition as the mental models mediating between discourse and society, and society as both local interactional settings and broader social structures such as institutions, classes, and cultures. These dimensions interact in bottom-up and top-down ways. Analysis begins with semantic macrostructures, which represent global meanings, themes, or topics that reflect the content of discourses and are often consciously constructed by speakers, typically visible in titles, summaries, or headlines.

Relevance in discourse is subjective and context-dependent, influenced by the speaker's goals, experiences, and knowledge \citep{van_dijk_principles_1993}.

The Discourse-Historical Approach enables a multi-layered critique of political communication, particularly suited to analysing mnemonic and ideological narratives, such as those produced by PiS. The approach distinguishes three complementary types of critique: text-immanent, sociodiagnostic, and prospective. The text-immanent critique examines internal inconsistencies, contradictions, and ambiguities within the discourse itself. The sociodiagnostic critique situates the text within broader socio-political contexts, uncovering persuasive strategies and mechanisms of influence. The prospective critique, in turn, aims to enhance discursive practices by offering guidelines for more inclusive communication \citep{tracy_critical_2015}.

Digital Ethnography complements the textual analysis by exploring how people interact with and interpret political memory online. Observing postings, comments, reposts, hashtags, and memes reveals how memory narratives are shared, challenged, or strengthened by online communities. This method captures the emotional, visual, and interactive aspects of memory in digital spaces. As digital media becomes central to political life, communication is no longer one-way; users actively interpret, remix, and respond to messages. Digital Ethnography helps study both official messages and grassroots reactions on platforms like X, showing how people affirm, question, or reshape memory narratives during campaigns \citep{matthes_digital_2017}.

\subsection{Data Analysis Strategy}

Tweets were first translated using an LLM and subsequently post-processed in 4Cat. Data analysis proceeded in two interrelated phases, enabling the construction of a corpus of political communication suitable for both quantitative (e.g., word frequency, hashtag usage) and qualitative (e.g., narrative shifts in memory culture, filtering down) analysis. Quantitative analysis focused primarily on hashtags and word frequency. Subsequently, tweets exhibiting notable content or patterns were manually selected for qualitative examination and analysed in context (see Chapters \hyperref[sec:social_media]{4.3} and \hyperref[Chap:Discussion]{5}).

In applying DHA to these social media messages, the analysis focused on shifts in rhetorical strategies, with particular attention to the final month of each election campaign, a period of intensified digital communication. The approach connects macro- and meso-level contextualisation to micro-level text analysis. It involves two main steps: entry-level thematic analysis, which deconstructs text content and assigns it to discourses using discourse topics \citep{van_dijk_principles_1993}, and in-depth analysis, which examines genre, macrostructure, argumentation, identity construction, and linguistic realisation, guided by research questions \citep{reisigl_discourse_2001}.

Scraped posts were retrieved through a keyword and phrase search strategy. Given the high degree of inflexion in Polish, stemming proved impractical; for instance, the word "pamięć" (memory) and its derivatives appear in numerous forms. A lemmatisation approach for this issue is plausible, although research into this topic is limited \citep{krasnowska2019empirical}; an open-source lemmatiser for the Polish language exists \citep{gawinecki_lemmatizer-pl}. Unfortunately, the implementation is also very resource-intensive, and a significant amount of programming would have been required to get the lemmatiser to run, which would also have been far beyond my capabilities. Having discovered the high error rate in this method, I switched to using only English, with stems derived from translated equivalents (e.g., "rememb" for "remember"). This dual-language approach facilitated a systematic retrieval of discourse about memory politics and legitimation practices \citep{karwatowski_context_2022}.

Together with Critical Discourse Analysis, this approach is flexible and iterative, focusing on text and hashtags as tools for building collective identity and historical meaning. In short, studying mnemonic narratives today requires methods that understand the emotional and symbolic power of memory, especially in online spaces where creators and audiences overlap \citep{matthes_digital_2017}.

\section{Limitations and Ethical Considerations}

This research integrates Memory Studies, Political Communication, and Digital Ethnography, drawing on interdisciplinary fields. It argues that digital memory systems should move beyond simple computational metaphors, such as "storage" and "retrieval", to embrace the complex, dynamic, and socially situated nature of human memory, informed by historical and cultural mnemonic practices \citep{van_house_technologies_2008}.

The thesis also critically addresses the risk of digital utopianism, cautioning that digital archives and infrastructures may perpetuate inequalities and marginalise minority voices if not designed with social and political awareness \citep{van_house_technologies_2008}. Methodologically, the research employs an inductive digital ethnography approach, which allows for flexible and responsive engagement with the rapidly changing digital political discourse, while carefully documenting coding and thematic decisions to mitigate confirmation bias \citep{day_finding_2019}.

Ethical considerations are central: only publicly accessible data is archived minimally and securely, considering privacy and platform policies, as well as recognising the nuanced expectations of consent and audience in digital contexts. The researcher's role involves continuous reflexivity, particularly because digital ethnography blurs traditional boundaries between public and private, requiring attention to the researcher's presence and influence in online spaces \citep{matthes_digital_2017}.

Initially, the research proposal and ethical approval included plans to analyse Facebook. However, social media platforms, particularly Facebook, have been linked to a decline in academic performance among students. Consequently, Facebook was removed from the analysis, as its algorithm and frequent updates introduced more distractions than valid data \citep{minds2lead2025meta}. Instead, the thesis focuses on X, which offered concise updates and more manageable data that were better suited for academic purposes. Recent changes at Meta in February 2025 further supported this decision \citep{kasanmascheff2025meta}.

Previously, the option of using Python for analysing Facebook was explored, which is commonly recommended for data handling. This proved too difficult for the purposes due to the researchers' limited programming experience \citep{minds2lead2025meta}. These experiences illustrate that while advanced tools can be powerful, they require a certain level of technical expertise to be effective. Furthermore, the option of automatically classifying the collected data using the LLM was explored, but this proved to be very resource-intensive in terms of both power consumption and time.

 Overall, prioritising X over Facebook and acknowledging the researchers' limited programming knowledge allowed this thesis to maintain productivity and gather more reliable data without being hindered by technical or platform-related challenges. In conclusion, this study's interdisciplinary and ethically sensitive methodology is designed to capture the complex interplay between digital technology, political memory, and social discourse, offering nuanced insights into how collective memory is constructed and contested in digital environments.

\vfill