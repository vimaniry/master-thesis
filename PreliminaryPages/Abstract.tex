\section*{Abstract}

In contemporary Poland, the politics of memory have emerged as a key site of national identity formation and political legitimacy. The Law and Justice Party (PiS) has emerged as a pivotal political force in shaping historical narratives, rendering collective memory a tool of governance, a means of exclusion, and a device for populist mobilisation. While attempts to post-communist European memory politics have continued to rise, comparative, digital, and discourse analyses of how mnemonic narratives evolve within the same political party over time remain rare. This thesis examines how PiS has served as a "mnemonic warrior" during the 2015 and 2023 election campaigns by analysing how narratives of victimhood, betrayal, and national purity are constructed, reinterpreted, and disseminated through both traditional and new media. Using a mixed-methods design combining Critical Discourse Analysis (CDA) and Digital Ethnography, the study examines party manifestos, media coverage, and social media accounts. The research employs the Discourse-Historical Approach to CDA and uses tools such as 4CAT and Zeeschuimer for gathering and analysing online material. Through examination, clear consistency in PiS's practices of memory is identifiable, and these hinge upon selective victimhood, elite delegitimisation, and historical revisionism. But the 2023 campaign demonstrates an intensification of digital approaches, particularly on X (formerly Twitter), where previous stories are re-authored in emotive, moralising, and exclusionary narratives aimed at constructing a closed memory community. The study contributes to the understanding of right-wing populist actors weaponising memory to advance polarisation and suppress pluralism. It highlights the necessity of temporally comparative and interdisciplinary study of memory politics, particularly in relation to increasing influence.

\vfill

\noindent\textbf{Keywords:} Collective Memory, Critical Discourse Analysis (CDA), Digital Ethnography, Law and Justice Party (PiS), Memory Politics, Mnemonic Warrior, Othering, Populism, Victimhood

\newpage\null\thispagestyle{empty}\newpage

\section*{Streszczenie}
We współczesnej Polsce polityka pamięci stała się kluczowym elementem kształtowania tożsamości narodowej i legitymizacji politycznej. Partia Prawo i Sprawiedliwość (PiS) stała się kluczową siłą polityczną w kształtowaniu narracji historycznych, czyniąc pamięć zbiorową narzędziem rządzenia, środkiem wykluczenia i instrumentem mobilizacji populistycznej. Podczas gdy próby postkomunistycznej europejskiej polityki pamięci nadal nasilają się, analizy porównawcze, cyfrowe i dyskursywne dotyczące ewolucji narracji mnemonicznych w ramach tej samej partii politycznej w miarę upływu czasu pozostają rzadkością. W niniejszej pracy przeanalizowano, w jaki sposób PiS pełniło rolę „wojownika pamięci” podczas kampanii wyborczych w 2015 i 2023 r., analizując sposób konstruowania, reinterpretacji i rozpowszechniania narracji o ofiarach, zdradzie i czystości narodowej zarówno w mediach tradycyjnych, jak i nowych. Wykorzystując metodę mieszaną, łączącą krytyczną analizę dyskursu (CDA) i etnografię cyfrową, badanie analizuje manifesty partyjne, relacje mediów i konta w mediach społecznościowych. W badaniach zastosowano podejście dyskursowo-historyczne do CDA oraz narzędzia takie jak 4CAT i Zeeschuimer do gromadzenia i analizowania materiałów internetowych. W wyniku analizy można dostrzec wyraźną spójność w praktykach PiS dotyczących pamięci, które opierają się na selektywnym traktowaniu ofiar, delegitymizacji elit i rewizjonizmie historycznym. Jednak kampania z 2023 r. wykazuje intensyfikację działań w mediach cyfrowych, zwłaszcza na platformie X (dawniej Twitter), gdzie wcześniejsze historie są przerabiane na emocjonalne, moralizatorskie i wykluczające narracje mające na celu stworzenie zamkniętej społeczności pamięci. Badanie przyczynia się do zrozumienia, w jaki sposób prawicowi populiści wykorzystują pamięć do pogłębiania polaryzacji i tłumienia pluralizmu. Podkreśla ono konieczność prowadzenia porównawczych i interdyscyplinarnych badań nad polityką pamięci, szczególnie w kontekście jej rosnącego wpływu.

\vfill

\noindent\textbf{Słowa kluczowe:} Etnografia cyfrowa, Inność, Krytyczna analiza dyskursu (CDA), Ofiarowość, Pamięć zbiorowa, Polityka pamięci, Populizm, Prawo i Sprawiedliwość (PiS), Wojownik Mnemonic