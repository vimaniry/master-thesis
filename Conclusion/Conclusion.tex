\chapter[Conclusion]{Conclusion}
\label{Chap:Conclusion}

This study demonstrates that the collective memory politics of Law and Justice (PiS) during its 2015 to 2023 election campaigns are marked by both continuity and change. Certain mnemonic tropes, othering, victimhood, nationalism, Catholicism, and commemoration, remain central and stable over time, forming the ideological backbone of PiS’s project of national identity construction \citep{zuk_anti-semitic_2023}. These elements provide a coherent narrative framework that consistently divides society into "true Poles" and ideological or foreign "others" \citep{kapralski_jews_2017}. Such continuity underscores the durability of populist identity politics that are deeply anchored in emotionally charged historical narratives \citep{couperus_memory_2023}.

The resulting mythologised narratives consolidate a Polish martyrological identity, portraying Poland as a noble victim of Nazi and Soviet aggression while often overlooking historical complexities and uncomfortable truths regarding Polish complicity \citep{meijen_populist_2024}. The myth of a "heroic and noble Poland", devoid of "black spots" in its history, simplifies the past to protect nationalist innocence and suppress pluralistic or critical interpretations \citep{zuk_anti-semitic_2023}. At the heart of this mnemonic conflict lies a contestation over Poland’s recent past, where competing narratives of heroism, victimhood, and marginalisation expose deep social fractures \citep{piotrowski_between_2010}. 

Thus, PiS’s narrative establishes a persistent "us versus them" cleavage. The party positions itself as the sole defender of Polish sovereignty, identity, and traditional values, setting itself against other opposition parties and supranational entities. By invoking historical memory and cultural continuity, PiS mobilises support and portrays Poland as a resilient nation resisting both external domination and internal decay \citep{woycicka_mnemonic_2024}. Across electoral cycles, memory serves both as a tool for national unity and as a means of ideological legitimation. The 2023 cycle reflects more pervasive and organised memory politics as a method of ideological control \citep{rybicki_2025_16933320}, a concern echoed by liberal media and opposition actors \citep{woycicka_mnemonic_2024}.

Nevertheless, within this continuity lies a significant change in emphasis, intensity, and medium. While the Smoleńsk tragedy operated as a constitutive myth in the 2015 campaign, ritualised in commemorations and heavily featured in political discourse, by 2023 its explicit centrality had diminished \citep{rybicki_2025_16933320}. Its symbolic weight, however, persisted as a background reference point, embedded within broader narratives of martyrdom and Polish victimhood \citep{woycicka_mnemonic_2024}. More decisive is the transformation in how PiS deploys memory politics: the digital sphere, and especially social media, has become a primary stage for advancing mnemonic narratives \citep{jamieson_theorising_2002}. On these platforms, repetition, speed, and emotional immediacy enable PiS to normalise exclusionary rhetoric and disseminate nationalistic frames with a far greater reach. The tone has also grown more aggressive, leveraging the affordances of digital media to intensify polarisation and diminish the legitimacy of dissent \citep{meijen_populist_2024}.

Politically, these memory practices serve a dual purpose. First, they reinforce PiS’s claim to domestic legitimacy, presenting the party as the sole custodian of Poland’s historical truth and moral identity \citep{korycki_memory_2017}. Second, they function to delegitimize opponents by collapsing a wide spectrum of adversaries,  Civic Platform (PO), Russia, Germany, and the EU, into a single symbolic enemy \citep{mazzini_three-dimensional_2018}. This strategy of amalgamation simplifies political conflicts into a stark moral binary, sustaining a siege mentality and fortifying in-group cohesion \citep{bernhard_notitle_2014}. By fusing disparate threats into one melting pot of hostile "others", PiS effectively weaponizes memory politics as a populist tool of mobilization and governance \citep{woycicka_mnemonic_2024}.

The findings also highlight broader implications for the study of populism and collective memory. PiS’s ability to institutionalize mnemonic politics, through commemorations, legislative measures, and control of public media, demonstrates how memory regimes can be transformed into enduring instruments of political legitimacy \citep{zuk_anti-semitic_2023}. In doing so, the party not only shapes historical consciousness but also actively reconfigures national identity in ways that exclude pluralistic interpretations and demonize perceived outsiders \citep{mazzini_three-dimensional_2018}.

Future research should build on the Large Language Model dataset generated for this thesis, which provides a promising methodological tool for tracing the circulation and transformation of mnemonic narratives across digital platforms. One particularly intriguing avenue is to explore how PiS’s strategy of collapsing multiple adversaries into one symbolic "other" operates in different contexts and how this shapes both domestic politics and Poland’s international positioning \citep{rybicki_2025_16933320}. This would allow scholars to better understand not only the tactical use of memory politics but also the deeper structural role of digital media in enabling populist parties to entrench exclusive and antagonistic visions of national identity. Strengthening historical literacy remains an urgent task, as greater awareness of the complexity of the past can empower both policymakers and citizens to resist oversimplified, exclusionary narratives and foster a more pluralistic, democratic debate about history and identity \citep{mumford_parallels_2015}.
 