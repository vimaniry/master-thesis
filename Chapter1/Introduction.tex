\chapter[Introduction]{Introduction}
\label{Chap:Introduction}

Since 2005, Poland's Law and Justice Party (PiS) has actively worked to reshape the country's memory. PiS has used laws, institutions, and symbolic actions to turn memory into a tool of political power. This approach simplifies history by dividing people into heroes and traitors, serving a specific ideological agenda \citep{lazor_memory_2016}. Under PiS, memory becomes a space where the state’s version of history is enforced, alternative views are silenced, and national identity is reshaped around a narrative of victimhood \citep{stanczyk_commemorating_2014}.

These dynamics can be explained by the theory of "mnemonic narratives", which are selective and structured representations of the past that are shaped and shared within political, institutional, and cultural contexts \citep{assmann_transnational_2014}. Memory is not fixed; it is constantly reshaped in political contexts and serves both as a performance and as a means of setting norms \citep{olick_social_1998}. Mnemonic narratives are not just about recalling the past; they are tools of power that shape group identity and justify authority \citep{bernhard_notitle_2014}.

Within this context, the Law and Justice party (PiS) operates as a "mnemonic warrior", a political actor that strategically uses memory to assert dominance over national narratives. Rather than engaging in an honest reflection on the past, PiS treats memory politics as a battleground where competing actors fight to impose singular, unchallenged interpretations of history. This more exclusionary and combative approach reflects what some scholars refer to as a "moral reconstruction" of the past, in which complex histories are reduced to simplistic dichotomies of heroes and traitors \citep{brants_transitional_2013}.

A clear example of this strategy is PiS’s 2018 "memory law" \citep*{sejm_rzeczypospolitej_polskiej_obwieszczenie_2018}, which criminalised public claims of Polish involvement in the Holocaust. This legislation illustrates how legal tools are used to enforce a state-sanctioned version of history, effectively institutionalising the nation's victimhood \citep{verovsek_caught_2021}. PiS has also heavily invested in historical institutions, museums, and the education system, utilising them to promote a sanitised version of the past aligned with its political objectives \citep{meijen_populist_2024}.

Simultaneously to promoting a tailored version of the past, this mnemonic strategy works as a tool of othering,  portraying liberal elites as morally compromised beneficiaries. PiS positions itself not only as a reformer but as a founder of the "Fourth Republic", a moral and political break from the post-Round Table compromise that marked the Third Republic. This ideological framework is used to delegitimise the liberal elites of the post-1989 transformation \citep{pozarlik_momentarily_2022}.

In the European memory landscape, post-communist states like Poland engage in what can be termed as "mnemonic competition", challenging Western Europe's Holocaust-centred memory by foregrounding their own experiences of communist oppression \citep{subotic_political_2018}. Whereas Western discourse often foregrounds Nazi crimes and the Jewish experience, these actors assert that suffering under communist regimes is equally significant and deserving of recognition. This creates asymmetries that are grounded in a narrative that portrays Poland as a perpetually victimised nation, oppressed by foreign powers, ignored by Western narratives, and betrayed by domestic elites. By constructing a transhistorical sense of victimhood that links events such as Katyń and the Smoleńsk air crash, PiS frames Poland as morally pure and constantly under threat \citep{fredheim_memory_2014}. These events are not only commemorated but also sacralised, integrated into a symbolic narrative in which suffering becomes a defining trait of Polish identity \citep{riedel_tri-marium_2022}.

Domestically, PiS extends this narrative by framing betrayal as both external and internal. Urban liberals, elites, and minority activists are seen as working with enemies and threatening the nation’s morals and culture \citep{piotrowski_between_2010}. These groups are often associated with historical stigmas, such as alleged Polish complicity in wartime atrocities, and are depicted as symbols of moral decay \citep{kucia_europeanization_2016}. Through this, PiS establishes rigid moral boundaries, allowing only specific, ideologically aligned versions of the past to be publicly acknowledged \citep{grabowski_memory_2018}.

By legally restricting memory, excluding alternative narratives, and using symbolic action, PiS suppresses pluralism and constructs a "closed memory community". In this way, PiS’s memory politics do not merely alter how the past is perceived; they actively reshape it to promote a strict, controlling notion of national identity and independence \citep{assmann_transnational_2014}.

This thesis examines how PiS’s role as a mnemonic warrior transforms memory into a political tool. It explores how narratives of trauma, betrayal, and national purity are framed to generate political legitimacy, shifting memory from a site of debate to a mechanism of domination, within their election campaigns in 2015 and 2023.

\section{Research Questions}

\begin{displayquote}
\textbf{What are the factors for change and continuity in the collective memory trends adopted by PiS from the 2015 election campaign to those in 2023?}

\vspace{1cm}

\textbf{In what ways and for what political purposes does PiS utilise memory politics via the digital sphere, particularly social media?}
\end{displayquote}

\section{Research Relevance}

This thesis is part of the broader field of collective memory research, focusing specifically on the Law and Justice party (PiS) and its role in shaping collective memory.

By examining the Law and Justice party's (PiS) strategic deployment of memory politics in the 2015 and 2023 campaigns, this thesis investigates how historical narratives are employed to legitimise rule, construct nationhood, and appeal to populist sentiment. This work contributes to ongoing debates on the political instrumentalisation of the past in modern Europe, particularly in the context of far-right populism \citep{couperus_memory_2023}.

Recent scholarship has demonstrated that the concepts of "collective memory" and "memory regimes" are often used interchangeably in political discourse, thereby creating confusion and reducing analytical precision. This interchange fails to grasp the broader power dynamics, media forms, and institutionalised forces that construct and reshape memory narratives. A more intricate methodology is necessary, one that considers the interplay among the official narrative, expert knowledge, and media in the process of memory construction \citep{dujisin_reassessing_2024}.

Despite growing academic interest in memory politics, the topic has been unevenly focused on Western European and Russian cases, often marginalising the distinctive mnemonic dynamics of Central and Eastern Europe \citep{berger_politics_2021}. This imbalance in research overlooks the multifaceted way post-communist nations, such as Poland, navigate contests of memory influenced by historical occupation, political transformation, and European Union integration \citep{malksoo_memory_2009}. \pagebreak A more urgent call is needed for comparative and interdisciplinary research that brings Central and Eastern European viewpoints to the forefront of the broader field of European memory studies. 

Social media-associated sites such as X (formerly Twitter) have emerged as central locations for the creation and sharing of collective memory. However, research into how these sites are used strategically by political parties, such as PiS, to construct collective memory and historical narratives for electoral and ideological purposes is limited \citep{bresco_de_luna_end_2017}. Although researchers have examined the transnational flow of memory, they have given comparatively little consideration to how digital technologies, especially social media, not only accelerate this process but also transform affective, political, and ethical entanglements with historical narratives \citep{assmann_transnational_2014}. Methodologically, memory studies as a field have yet to incorporate approaches from media and communication studies. There is an urgent need to understand the mechanisms by which memory narratives are facilitated, received, and contested in hybrid media environments, particularly on social media platforms that enable the dissemination of populist rhetoric and emotive arguments about the past \citep{kansteiner_finding_2002}.

Empirical evidence suggests that comparative or transnational studies on memory politics in post-communist Europe are scarce \citep{bresco_de_luna_end_2017}. The employed mixed-methods approach is increasingly valued for its ability to explain the complex and sometimes contradictory ways in which memory influences populist mobilisation \citep{couperus_memory_2023}. This involves studying how crises, symbolic actions, and media technologies shape the appeal and impact of populist memory policies \citep{wodak_politics_2015}. The growing use of forceful tools, such as state-backed versions of history and information warfare, poses a significant challenge to traditional historical theory and critical historiography. Here, how the past is imaginatively constructed and temporally placed ethically is not merely an issue of ideology; it has a powerful influence on the formation of identities and fuels political contestation \citep{korycki_memory_2017}.

Additionally, the temporal dimension is crucial: populist parties are ambiguous and evolve in response to external conditions. Since no universal strategy exists to manage populism, responses must be context sensitive and calibrated to specific conditions \citep{crum_populist_2022}. By examining PiS's mnemonic strategy across two electoral cycles, this study offers empirical insight into the temporal transformation of populist memory politics in Poland, thereby enriching the comparative scope of memory studies.

This thesis combines the disciplines of memory studies, political science, populism studies and political sociology to examine how the PiS party uses constructed historical narratives to gain political legitimacy. It focuses primarily on digital media, particularly social media, as a key space for populist memory practices. Using this cross-disciplinary and mixed-method approach, the thesis provides new insights by tracking memory narratives across election periods and different media environments in Poland. While situated within the broader field of collective memory studies, this thesis contributes to a more focused body of work addressing the strategic use of historical narratives by right-wing populist actors in post-communist Europe.

\section{Thesis Structure}

Chapter Two presents the theoretical framework, incorporating relevant concepts such as othering, victimhood, populism, and collective memory. Additionally, it examines the political aspects of memory and the idea of the mnemonic warrior.

Chapter Three traces the methodological approach, describing the research design, data collection methods, and analytical methods, with a focus on Critical Discourse Analysis and Digital Ethnography. 
Chapter Four presents an empirical analysis, discussing data drawn from news reports, social media postings and electoral manifestos. The results are then placed within the theoretical context in Chapter Five. The final chapter will conclude the thesis by bringing together the principal findings, situating them within broader academic discussions, and identifying implications for future research.
