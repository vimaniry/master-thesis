\chapter[Discussion]{Discussion}
\label{Chap:Discussion}
This chapter examines how PiS mobilises historical myths, collective trauma, and selective interpretations of national identity to consolidate political authority, delegitimise opposition, and shape public perceptions of Poland’s past and present. The discussion situates PiS within broader Central European trends of democratic backsliding, highlighting the interplay between populist rhetoric, nationalist ideology, and mnemonic governance in constructing a dominant vision of the nation.

\section{Interpretation of Results}

A central element of PiS's political discourse is the revival of historical paradigms that portray Poland as a sovereign, morally superior power resisting external domination. One prominent example is the idea of Intermarium, initially envisioned by Józef Piłsudski, which positions Poland as a pre-communist regional power countering threats from Moscow and Berlin \citep{riedel_tri-marium_2022}. PiS's election platforms actively invoke this legacy, depicting Poland as a moral and geopolitical leader in the post-Soviet space and calling for the unity of Central and Eastern European states to defend sovereignty against foreign pressures \citep{pis_program_2023}.

The 2010 Smoleńsk air crash is the most prominent example of PiS’s instrumentalisation of victimhood, transformed into a sacrosanct national trauma comparable to the Katyń massacre. This event reinforces a narrative of Soviet betrayal and eternal grief \citep{fredheim_memory_2014}. Religious and sacrificial language is used to position PiS as the rightful heir to Poland’s heroic dead and defender of national pride \citep{ksiazek_smolensk_2018}. PiS rejects the "pedagogy of shame" associated with earlier acknowledgement of Polish complicity in WWII atrocities, promoting instead a discourse of Polish innocent victimhood \citep{woycicka_mnemonic_2024}.

Catholicism is tightly woven into these historical and national narratives, framing Poland’s identity within a divine mission that legitimises assertive security policies against historic threats from Russia and Germany \citep{lazor_memory_2016}. 

\pagebreak

Public commemorations, including Katyń remembrance and national anniversaries such as the Independence March, blend pride and grief while reinforcing anxieties about sovereignty \citep{pis_program_2023}.

Katyń functions as a mnemonic hotspot, where patriotism is equated with resistance to Europeanization and perceived treason \citep{fredheim_memory_2014}. National anniversaries similarly symbolise Poland’s rebirth and struggle against foreign domination \citep{kobierecki_opportunistic_2022}.

Communism is portrayed as fundamentally alien and illegitimate, often compared to Nazism and sometimes infused with antisemitic undertones \citep{korycki_memory_2017}. This discourse constructs a narrative of victimhood, legitimising PiS’s opposition to post-communist elites. Policymakers frequently employ simplified or mythologised historical analogies to justify systemic reforms. The post-1989 democratic transition in Poland and Hungary is reinterpreted as a tale of betrayal and moral decline. This narrative is used to legitimise initiatives such as the "Fourth Republic" and a revived ethno-organic national state. Similarly, the 1989 Round Table negotiations are depicted as illegitimate by populist actors \citep{pozarlik_momentarily_2022}.

Kaczyński and PiS argue that since 1989, an "old-new elite" has monopolised political, economic, and cultural power in Poland, consisting of an alliance between post-communist and liberal post-Solidarity forces. According to the party’s 2014 program, this negative "pseudo-elite" is beholden to foreign interests, exhibits a "distaste . . . for the [Polish] state," and maintains a "suspicious attitude towards the nation" \citep{bill_counter-elite_2022}. Kaczyński contends that liberal civil society, media, and cultural elites share this outlook, often rejecting the very concept of the nation. These groups, he claims, submissively emulate Western trends in ways that betray authentic Polish interests and values. The "pseudo-elite" must therefore be replaced by a "counter-elite," a set of PiS-aligned groups dedicated to promoting national interests. In this sense, PiS’s populism is a form of "counter-elite" populism; it is not inherently anti-elite. Instead, it defines a fundamental conflict between a false elite and the people, who are represented by the true (counter-)elite \citep{bill_counter-elite_2022}.

This framing is echoed in recent statements by PiS leaders. For example, Minister Czarnek asserted that if Tusk had been in power during Russia’s attack on Ukraine, genocidal forces would have reached Poland, and eastern Poland would have continued to lack investment and infrastructure \citep{pisorgpl2023p}. Similarly, Kaczyński warned that a return to Tusk’s leadership would revive past chaos, plundering, and subordination to Brussels and Germany, emphasising that PiS governs independently as a free and sovereign nation \citep{pisorgpl2023q}. Kaczyński further claimed that Tusk’s party is "external," effectively German, seeks to undermine democracy, employs thuggish methods, and relies solely on lies \citep{pisorgpl2023r}.

The legacy of Solidarność, while not always explicitly invoked, remains central in legitimising Poland’s post-1989 identity as an anti-communist bastion \citep{piotrowski_between_2010}. PiS construes Solidarność as a fixed symbol of conservative nationalism, truncating its original, inclusive social movement origins \citep{piotrowski_between_2010}. Earlier governments had presented Solidarność in a more pluralistic manner, emphasising inclusivity and pro-European values \citep{woycicka_mnemonic_2024}. This selective narrative sidelines progressive and minority voices, consolidating conservative interpretations of history and limiting space for dissent \citep{assmann_transnational_2014}. PiS’s politics of memory deliberately reinforce a singular vision of national history based on heroism, martyrdom, and victimhood, excluding alternative perspectives \citep{mazzini_three-dimensional_2018}. Commemoration is repoliticized, portraying Poland as a perennial victim of communism and moral decay under post-1989 democratic elites, whom the party frames as traitors \citep{lazor_memory_2016}. PiS’s mnemonic warfare strategy is not merely tactical; it is integral to its broader ideological project of building a "Fourth Republic," framing memory as an existential imperative and moral obligation necessary for national renewal.

Their mythologised narratives also serve contemporary political purposes, particularly in delegitimising opposition. The Smoleńsk discourse frames Civic Platform and Donald Tusk as betrayers of Polish sovereignty \citep{ksiazek_smolensk_2018}. By selectively highlighting Solidarność’s conservative aspects and marginalising liberal and minority perspectives, PiS consolidates its nationalist ideology while minimising opportunities for critical reflection or dissent \citep{piotrowski_between_2010}. Solidarność is made central to conservative nationalism through the exclusion of liberal and minority voices \citep{lazor_memory_2016}.

Their mythologised narratives also serve contemporary political purposes, particularly in delegitimising opposition. The Smoleńsk discourse frames Civic Platform and Donald Tusk as betrayers of Polish sovereignty \citep{ksiazek_smolensk_2018}. By selectively highlighting Solidarność’s conservative aspects and marginalising liberal and minority perspectives, PiS consolidates its nationalist ideology while minimising opportunities for critical reflection or dissent \citep{piotrowski_between_2010}. Solidarność is made central to conservative nationalism through the exclusion of liberal and minority voices \citep{lazor_memory_2016}.

From 2015 to 2023, PiS increasingly positioned itself as reclaiming Poland’s national sovereignty and dignity from foreign domination, combating EU federalism and political homogenisation, and advancing military and diplomatic strength to protect regional stability \citep{pis_program_2023}. It framed the EU as a neo-imperial hegemon and cast itself as the leader of a coalition resisting ideological domination from Brussels \citep{riedel_tri-marium_2022}. 

\pagebreak

Political opponents were othered as foreign elites hostile to "true Poles," legitimising their exclusion and the concentration of power \citep{siedlicka_pis_2015}. These discourses supported the consolidation of political authority \citep{wronski_pis_2015}.

\section{Relating to Literature}

Since 2015, PiS has evolved from its earlier "memory warrior" phase (2005–2007) into a more radical form. The party acts as a memory excluder and implements systematic mnemonic politics across domestic and foreign policy spheres \citep{mazzini_three-dimensional_2018}. Guided by a "politics of certainty," PiS frames itself as the sole guardian of authentic national identity. It also positions itself as the revealer of hidden truths about a betrayed political transition \citep{dujisin_reassessing_2024}. Following 2010, and especially after the Smoleńsk tragedy, PiS militarised memory through the takeover of public institutions. This memory regime envisions Poland as a heroic nation under siege. It legitimises illiberal governance and disenfranchises opposition \citep{woycicka_mnemonic_2024}.

Poland’s post-EU accession period initially witnessed a permissive, inclusive memory culture promoted by civil society and NGOs \citep{stanczyk_commemorating_2014}. PiS’s return to government marked a shift toward a closed, exclusionary model of memory governance, creating a deep divide between liberal and illiberal historical narratives \citep{stanczyk_commemorating_2014}. The party constructs an existential political myth of struggle between "true Poles" and corrupt post-communist elites. This myth instrumentalises collective memory for political purposes \citep{korycki_memory_2017}.

Institutional reforms support memory work beyond nationalist narratives, recuperating concepts such as "Piast Poland" and the "recovered territories" to cement German guilt and Polish victimhood \citep{langenbacher_twenty-first_2008}. Similarly, the party frames its refusal to sign international climate treaties as a defence of Polish sovereignty against global Others, reinforcing nationalist identity through climate change denial \citep{ulanowski_pis_2015}.

The EU is perceived as remote and abstract, failing to foster a habitual and emotional attachment \citep{jamieson_theorising_2002}. Populist and nationalist movements, such as PiS, exploit this gap to reject supranational identities \citep{jamieson_theorising_2002}.

PIS´s strategy exemplifies the ideocratic turn in Central Europe. Democratic systems are transformed into illiberal, majoritarian regimes through a combination of populist rhetoric, nationalist memory politics, and ideological reinterpretation of history \citep{rupnik_crisis_2018}. The outcome is a mnemonic order in which dissent is treated as treason. Memory becomes ideology. The national "self" is continuously reinforced through the exclusion and vilification of the "other" \citep{mazzini_three-dimensional_2018}. These myths consolidate a Polish martyrological identity \citep{meijen_populist_2024}, through which Poland is consistently portrayed as the noble victim of Nazi and Soviet aggression \citep{gliszczynska_grabias_memory_2014}.

PiS populism is strongly tinged with nationalism \citep{brubaker_populism_2020}. Traditional Polish scholarship often defines nationalism narrowly in terms of pre-war National Democracy \citep{cordell_transformation_2015}. This thesis adopts a broader conception: loyalty to the nation takes precedence over other collective loyalties. Thus, the nation is the sole source of legitimate political authority \citep{jaskulowski_populist_2023}.


\section{Practical Impact}

From 2015 to 2023, the Law and Justice Party (PiS) consistently integrated its electoral programs into a nationalist-leaning ideology through the use of memory politics. Using populist rhetoric and anti-elitist messaging, PiS mobilises support by portraying ordinary Poles as reclaiming a history allegedly distorted by cosmopolitan elites, while minimising the experiences and agency of lower social classes \citep{jaskulowski_populist_2023}.

PiS presents history as a narrative of national heroes defending an ethnically homogeneous nation that is always a victim rather than an aggressor, using this memory politics as a tool of social pedagogy. Its nationalist ideology paradoxically frames national identity as deeply rooted yet insufficiently recognised by the populace, necessitating promotion by a self-proclaimed “authentic national elite” through education, media, and culture \citep{jaskulowski_populist_2023}. In countries such as Russia, Hungary, and Poland, conspiratorial narratives suggested that Western financial institutions and political elites, termed the “wrong elite”,  orchestrated the collapse of communism to plunder national resources and impose cultural dominance \citep{bergmann_strategic_2025}.

Elections are framed not as instruments of representation but as subordinate to the authority of elites interpreting the nation’s authentic identity; those who resist this vision face exclusion and pressure. Overall, PiS’s memory politics exemplifies its radical nationalist character, using populist strategies to foster national cohesion, cultural consolidation, and moral renewal \citep{jaskulowski_populist_2023}.
